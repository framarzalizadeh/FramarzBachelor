\chapter{Anhang}

\section{Verwendete Hilfsmittel}
In der Tabelle \ref{tab:tooling} sind die im Rahmen der Bearbeitung des Themas der \IthesisKindDE~verwendeten Werkzeuge und Hilfsmittel aufgelistet.

\begin{table}[h!]
\caption{Verwendete Hilfsmittel und Werkzeuge}
\begin{tabular}{|p{4cm}|p{10cm}|}
\hline 
\rowcolor{lightgray} Tool & Verwendung \\
\hline
\LaTeX & Textsatz- und Layout-Werkzeug zur Erstellung dieser Bachelorarbeit \\
\hline
Make.com & Implementierung der Chatbot-Workflows und Automatisierungslogik \\
\hline
Airtable & Zentrale cloudbasierte Datenhaltung für Kunden- und Rechnungsdaten \\
\hline
Google Docs & Vorlage und automatisierte Generierung von Rechnungsdokumenten \\
\hline
Google Drive & Strukturierte Ablage der erzeugten Rechnungen und Dokumente \\
\hline
Lovable Cloud / Supabase Edge Functions & Hosting der Web-Anwendung und serverlose Ausführung von Integrationslogik \\
\hline
React & Umsetzung der ClientHub-Webanwendung als Single-Page-Application \\
\hline
TypeScript & Typisierte Implementierung der Frontend-Logik \\
\hline
Tailwind CSS & Styling und Layout der Webanwendung \\
\hline
shadcn/ui & UI-Komponentenbibliothek für wiederverwendbare Interface-Elemente \\
\hline
Visual Studio Code & Entwicklungsumgebung für Frontend- und Edge-Function-Code \\
\hline
ChatGPT & Unterstützung bei Konzeption, Textstrukturierung und sprachlicher Überarbeitung \\
\hline
Perplexity AI & Rechercheunterstützung und Zusammenfassung technischer Sachverhalte \\
\hline
\end{tabular}
\label{tab:tooling}
\end{table}
