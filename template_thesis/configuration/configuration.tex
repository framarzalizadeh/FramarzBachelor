% !TEX root = ../thesis.tex
%
% configurations
%

% English Language support
% -> uncomment if needed
% Beta!
%\fullenglish{yes}
\fullenglish{no}

% text field
%-> replace supervisor names with correct ones
\firstSupervisor{Prof. Dr. Stefan Sarstedt}
\secondSupervisor{Prof. Dr. Ulrike Steffens}

% text field
%-> replace title with your thesis title
\thesisTitle{Automatisierte Rechnungs- und Dokumentenverarbeitung: Konzeption und Implementierung eines Chatbotbasierten  Systems zur Erstellung und Ablage von Rechnungen mit ergänzender Web-Anwendung für Verwaltung und Zugriff.}
\thesisTitleEN{Automated invoice and document processing: Design and implementation of a chatbot-based system for creating and storing invoices with a supplementary web application for management and access.}

% text field
%-> replace the key words with your own key words
\keywordsDE{Rechnungsautomatisierung, Dokumentenmanagement, Chatbot-Systeme, Cloud-Speicherung, Webanwendungen, Prozessdigitalisierung}
\keywordsEN{Invoice Automation, Document Management, Chatbot Systems, Cloud Storage, Web Applications, Process Digitalization}

% text field
%-> replace the text with a description of the thesis
\abstractDE{
Die manuelle Erstellung, Verwaltung und Ablage von Rechnungen stellt in vielen kleinen und mittleren Unternehmen eine erhebliche organisatorische und zeitliche Belastung dar. Insbesondere fehleranfällige Arbeitsschritte, redundante Dateneingaben sowie unstrukturierte Ablageprozesse erschweren ein effizientes Dokumentenmanagement. Ziel dieser Arbeit ist die Konzeption und Implementierung eines automatisierten Systems, das den gesamten Rechnungsprozess digitalisiert und durch einen Chatbot unterstützt. Die Datenerfassung erfolgt über ein dialogbasiertes Chatbot-System, das mithilfe der Plattform Make.com implementiert wurde und Kundendaten sowie Rechnungsinformationen strukturiert in Airtable speichert. Auf Basis dieser Daten werden Rechnungsdokumente automatisiert in Google Docs generiert und revisionssicher in einer hierarchischen Ordnerstruktur in Google Drive abgelegt. Ergänzend wurde eine moderne Webanwendung entwickelt, die als Verwaltungs- und Zugriffssystem dient und die Kundendaten sowie die generierten Dokumente übersichtlich darstellt. Die Web-App basiert auf React, TypeScript und Tailwind CSS und kommuniziert über serverlose Edge Functions mit der Airtable- und Google-Drive-API. Das Gesamtsystem zeigt, dass der Einsatz moderner Cloud-Technologien und automatisierter Workflows die Prozessqualität erhöht, Fehler reduziert und eine deutliche Effizienzsteigerung ermöglicht. Die Arbeit demonstriert das Potenzial kombinierter Chatbot- und Web-Technologien für eine zukunftsorientierte Digitalisierung administrativer Geschäftsprozesse.
}
\abstractEN{
The manual creation, management, and storage of invoices represents a significant organizational and time-consuming challenge for many small and medium-sized enterprises. Error-prone workflows, repeated data entry, and unstructured document storage limit the efficiency of traditional invoice processes. This thesis aims to design and implement an automated system that digitizes the entire invoice workflow and supports it through a chatbot-based interaction model. Data collection is performed via a dialog-driven chatbot built using Make.com, which stores customer and invoice information in Airtable in a structured manner. Based on this data, invoice documents are automatically generated using Google Docs and stored in a revision-proof, hierarchical folder structure in Google Drive. In addition, a modern web application was developed to provide administrative access and management features for customers and documents. The web application is built with React, TypeScript, and Tailwind CSS and communicates with Airtable and Google Drive through serverless edge functions. The system demonstrates that the use of modern cloud technologies and automated workflows can significantly improve process quality, reduce errors, and enhance operational efficiency. This thesis highlights the potential of combining chatbot technology and web-based interfaces to support the digital transformation of administrative business processes.
}

% text field
%-> replace john with your name
\thesisAuthor{Framarz Alizadeh}

% text field
%-> enter the submission date
\submissionDate{07. Juni 1954}

% switch - uncomment only one
%-> uncomment NDA or public
%\NDA{yes}
\NDA{no}

% switch - uncomment only one
%-> uncomment old standard cover or cover Corporate Design 2017
\Cover{CD2017}
%\Cover{CD2017NoLogo}
%\Cover{Std2018}
%\Cover{Std2018_green} 			% with green bar

% switch - uncomment only one
%-> uncomment to show list of figures or not
\ListOfFigures{yes}
%\ListOfFigures{no}

% switch - uncomment only one
%-> uncomment to show list of tables or not
\ListOfTables{yes}
%\ListOfTables{no}

% switch - uncomment only one
%-> uncomment to show list of accronyms or not
\ListOfAccronyms{yes}
%\ListOfAccronyms{no}

% switch - uncomment only one
%-> uncomment to show list of symbols or not
\ListOfSymbols{yes}
%\ListOfSymbols{no}

% switch - uncomment only one
%-> uncomment to show list of glossary entries or not
\Glossary{yes}
%\Glossary{no}

% switch - uncomment only one
%-> uncomment the study course you are in
\studycourse{INF_ITS}   % Informatik Technischer Systeme
%\studycourse{INF_AI}   % Angewandte Informatik
%\studycourse{INF_WI}   % Wirtschaftsinformatik
%\studycourse{INF_ECS}  % European Computer Science
%\studycourse{EMI_EI}   % Elektrotechnik und Informationstechnik
%\studycourse{EMI_REE}  % Regenerative Energiesysteme und Energiemanagement
%\studycourse{EMI_IE}   % Inforamation Engineering
%\studycourse{BMP}      % Mechanical Engineering
%\studycourse{BMP-hp}   % Mechanical Engineering - Internship Report
%\studycourse{BMT}      % Mechatronik
%\studycourse{BMT-st}   % Study / home assignment in BMT
%\studycourse{BMT-hp}   % Internship Report
%\studycourse{INF_MaI}      % Master Informatik
%\studycourse{EMI_MaA}      % Master Automatisierung
%\studycourse{EMI_MaICE}    % Master Information and Communication Engieering
%\studycourse{EMI_MaMS}     % Master Microelectronic Systems

\setcounter{secnumdepth}{2}
\setcounter{tocdepth}{2}