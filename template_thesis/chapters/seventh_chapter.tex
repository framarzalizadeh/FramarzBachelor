% first example chapter
% @author Thomas Lehmann
%
\chapter{Fazit und Ausblick}
\section{Zusammenfassung der Ergebnisse}
\label{sec:zusammenfassung-ergebnisse}

Ziel dieser Arbeit war es, einen niedrigschwelligen Ansatz zur teilautomatisierten Rechnungs- und Dokumentenverarbeitung für kleine Dienstleistungsunternehmen und Selbstständige zu entwickeln. Auf Basis der analysierten Ausgangssituation wurde ein prototypisches System entworfen, das einen Chatbot zur dialogbasierten Datenerfassung mit einer cloudbasierten Datenhaltung und einer ergänzenden Web-Anwendung zur Einsicht und Verwaltung von Kunden, Rechnungen und Dokumenten kombiniert. Im Fokus stand dabei die Unterstützung realer Arbeitsabläufe ohne Einführung komplexer Buchhaltungssoftware und mit möglichst geringem Anpassungsaufwand für die Nutzenden.

Im Rahmen der Konzeption wurden zentrale Use-Cases definiert, insbesondere die Erstellung von Rechnungen sowie die strukturierte Ablage administrativer Dokumente. Darauf aufbauend wurde eine Zielarchitektur entwickelt, die eine Integrationsplattform zur Orchestrierung der Prozesse, einen Messaging-Kanal für die Chatbot-Interaktion, ein cloudbasiertes Datenmodell für Kunden- und Rechnungsdaten sowie eine Web-Oberfläche für den Überblick und den Zugriff auf die erzeugten Dokumente umfasst. Die Dialogführung im Chatbot und die Struktur der zugrunde liegenden Daten wurden so gestaltet, dass eine schrittweise und nachvollziehbare Erfassung der relevanten Informationen ermöglicht wird.

Die prototypische Implementierung zeigt, dass sich mit vergleichsweise einfachen, standardisierten Cloud-Diensten ein durchgängiger End-to-End-Prozess realisieren lässt. Rechnungsdaten können dialoggeführt erfasst, in einer zentralen Datenbasis gespeichert und auf Basis eines Templates automatisiert zu \gls{ac:pdf}-Rechnungen generiert werden. Gleichzeitig werden Rechnungen und weitere Dokumente in einer konsistenten, nach Zeiträumen strukturierten Ordnerhierarchie abgelegt und über eine Web-Anwendung auffindbar gemacht. Die Evaluation anhand exemplarischer Szenarien ergab, dass die Kernfunktionen zur Rechnungserstellung, zur Ablage von Dokumenten und zur Anzeige der Daten grundsätzlich stabil ablaufen und die erwarteten Ergebnisse liefern.

Gleichzeitig wurden im Rahmen der Untersuchung auch Grenzen und Herausforderungen des prototypischen Ansatzes deutlich. Dazu zählen unter anderem der vergleichsweise lange Prozess der Rechnungserstellung bei der Neuanlage von Kunden aufgrund vieler einzelner Dialogschritte, eine nur teilweise abgedeckte Fehler- und Sonderfallbehandlung sowie eine Performance, die aus Nutzersicht zwar akzeptabel, aber nicht in allen Fällen optimal ist. Zudem zeigt sich, dass die Nutzung externer Cloud- und Integrationsdienste zwar die Entwicklung erleichtert, jedoch Abhängigkeiten und potenzielle Einschränkungen hinsichtlich Skalierung, Kosten und langfristiger Wartbarkeit mit sich bringt. Insgesamt belegen die Ergebnisse jedoch, dass der gewählte Ansatz grundsätzlich geeignet ist, Medienbrüche zu reduzieren und wiederkehrende manuelle Tätigkeiten in der Rechnungs- und Dokumentenverarbeitung zu unterstützen.

\section{Beantwortung der Forschungsfragen}
\label{sec:beantwortung-forschungsfragen}

In diesem Abschnitt werden die in Abschnitt~\ref{sec:forschungsfragen} formulierten Forschungsfragen auf Basis der in dieser Arbeit erzielten Ergebnisse beantwortet.

\subsection*{Forschungsfrage 1}
\textit{Wie lässt sich ein prototypisches System zur Erstellung und Ablage von Rechnungen und administrativen Dokumenten für kleine Dienstleistungsunternehmen und Selbstständige auf Basis gängiger cloudbasierter Werkzeuge konzipieren?}

Die Arbeit zeigt, dass ein solches System durch die Kombination eines Chatbots, einer Integrationsplattform, einer cloudbasierten Datenhaltung und einer ergänzenden Web-Anwendung realisiert werden kann. Die entwickelte Zielarchitektur nutzt einen bestehenden Kommunikationskanal für die Interaktion, eine zentrale Datenbasis für Kunden- und Rechnungsinformationen sowie standardisierte Cloud-Dienste für die automatisierte Dokumentenerzeugung und strukturierte Ablage. Damit konnte ein durchgängiger, auf wiederkehrende administrative Abläufe zugeschnittener End-to-End-Prozess konzipiert werden.

\subsection*{Forschungsfrage 2}
\textit{Inwieweit kann ein Chatbot, der an einen bestehenden Kommunikationskanal angebunden ist, die dialogbasierte Erfassung von Rechnungs- und Dokumentendaten so unterstützen, dass Medienbrüche reduziert und wiederkehrende manuelle Arbeitsschritte verringert werden?}

Die prototypische Umsetzung zeigt, dass ein Chatbot die Erfassung von Rechnungs- und Dokumentendaten grundsätzlich wirksam unterstützen kann, indem er Nutzende Schritt für Schritt durch den Prozess führt und die erfassten Informationen direkt an die nachgelagerten Automatisierungen weitergibt. Medienbrüche werden reduziert, da relevante Daten nicht mehr in separaten Office-Dokumenten oder E-Mails gepflegt, sondern im Rahmen des Dialogs strukturiert erfasst und verarbeitet werden. Wiederkehrende Tätigkeiten wie das Öffnen von Vorlagen, das manuelle Einfügen von Daten und das Speichern von Dateien werden weitgehend in automatisierte Abläufe verlagert, auch wenn die dialogbasierte Erfassung insbesondere bei der Neuanlage von Kunden noch relativ zeitintensiv ist.

\subsection*{Forschungsfrage 3}
\textit{Wie kann eine ergänzende Web-Anwendung gestaltet werden, die den Zugriff auf Kunden, Rechnungen und Dokumente strukturiert ermöglicht und die im Chat erfassten Daten für die weitere Verwaltung nutzbar macht?}

Die entwickelte Web-Anwendung zeigt, dass eine schlanke, auf Anzeige- und Filterfunktionen fokussierte Oberfläche ausreicht, um die im Chat erfassten Daten für die Verwaltung nutzbar zu machen. Durch die Darstellung von Kunden, Rechnungen und Dokumenten auf Basis einer gemeinsamen, konsistenten Datenstruktur wird ein schneller Überblick über den aktuellen Stand ermöglicht. Nutzende können erzeugte Rechnungen einsehen, heruntergeladene Dokumente auffinden und Zusammenhänge zwischen Kundendaten und zugehörigen Belegen nachvollziehen, ohne direkt mit den zugrunde liegenden Cloud-Diensten interagieren zu müssen.

\subsection*{Forschungsfrage 4}
\textit{Welche Grenzen, Herausforderungen und Verbesserungspotenziale zeigen sich beim praktischen Einsatz des prototypisch umgesetzten Systems im Hinblick auf Nutzbarkeit, Prozessdurchlaufzeiten und technische Robustheit?}

Die Evaluation macht deutlich, dass die Kernprozesse zur Rechnungserstellung und Dokumentenablage funktional stabil ablaufen, die wahrgenommenen Prozessdurchlaufzeiten aus Sicht Einzelner jedoch teilweise als lang empfunden werden. Insbesondere die Vielzahl einzelner Dialogschritte bei der Datenerfassung und die sequentielle Abarbeitung der Automatisierungsworkflows tragen zu spürbaren Wartezeiten bei. Hinsichtlich der technischen Robustheit zeigt sich, dass typische Eingabefehler abgefangen und Dialoge in vielen Fällen sinnvoll fortgeführt werden, komplexere Sonderfälle jedoch noch nicht vollständig abgedeckt sind. Daraus ergeben sich konkrete Verbesserungspotenziale, etwa in der Straffung der Dialogführung, der Erweiterung der Validierungslogik und der Optimierung der zugrunde liegenden Integrationsprozesse.

\section{Beitrag der Arbeit und praktische Implikationen}
\label{sec:beitrag-implikationen}

Die vorliegende Arbeit leistet einen Beitrag zur praxisnahen Digitalisierung administrativer Prozesse in kleinen Dienstleistungsunternehmen und bei Selbstständigen. Im Unterschied zu klassischen Ansätzen, die auf umfassende Buchhaltungs- oder \glspl{ac:erp}-Systeme setzen, verfolgt sie einen niedrigschwelligen, prototypischen Lösungsweg, der auf allgemein verfügbare Cloud-Dienste und einen bereits etablierten Kommunikationskanal aufbaut. Durch die Kombination eines Chatbots mit einer Integrationsplattform, einer cloudbasierten Datenhaltung und einer schlanken Web-Anwendung wird gezeigt, wie sich mit begrenztem technischen Aufwand ein durchgängiger Prozess zur Erstellung und Ablage von Rechnungen und Dokumenten realisieren lässt.

Aus wissenschaftlicher Perspektive besteht der Beitrag vor allem in der Konzeption und prototypischen Umsetzung eines integrierten Systems, das dialogbasierte Interaktion mit automatisierter Dokumentenerzeugung und strukturierter Ablage verbindet. Die Arbeit verdeutlicht, welche architektonischen Entscheidungen und Integrationsmuster sich für diesen Anwendungsfall eignen und wie bestehende Dienste so kombiniert werden können, dass Medienbrüche reduziert und wiederkehrende manuelle Tätigkeiten teilweise automatisiert werden. Darüber hinaus liefert die Evaluation Einblicke in die Grenzen eines solchen Ansatzes, etwa im Hinblick auf Prozessdurchlaufzeiten, Fehlerbehandlung und Abhängigkeiten von Drittanbietern, und schafft damit eine Grundlage für weiterführende Untersuchungen.

Für die Praxis ergeben sich mehrere Implikationen. Zum einen zeigt der Prototyp, dass kleine Unternehmen und Selbstständige ihre Rechnungs- und Dokumentenprozesse verbessern können, ohne sofort in komplexe oder kostspielige Spezialsoftware investieren zu müssen. Stattdessen können sie auf bereits genutzte Kommunikationskanäle und Cloud-Dienste zurückgreifen und diese gezielt zu einem sinnvoll orchestrierten Prozess verbinden. Zum anderen macht die Arbeit deutlich, dass insbesondere die Gestaltung der Dialoge und der Benutzerführung entscheidend dafür ist, ob ein solches System als Unterstützung und nicht als zusätzliche Hürde wahrgenommen wird. Die identifizierten Verbesserungspotenziale – etwa eine kompaktere Dialogführung, erweiterte Validierungen oder eine engere Verzahnung mit bestehenden buchhalterischen Abläufen – geben konkrete Hinweise darauf, wie vergleichbare Lösungen in der Praxis weiterentwickelt und schrittweise professionalisiert werden können.

\section{Ausblick und Weiterentwicklung}
\label{sec:ausblick-weiterentwicklung}

Die im Rahmen dieser Arbeit entwickelte Lösung zeigt, dass sich mit standardisierten Cloud-Diensten und einem Chatbot-basierten Ansatz ein durchgängiger Prozess zur Erstellung und Ablage von Rechnungen und administrativen Dokumenten realisieren lässt. Gleichzeitig wird deutlich, dass der Prototyp in mehreren Bereichen gezielt weiterentwickelt werden kann, um sowohl die Nutzbarkeit als auch die Automatisierungstiefe zu erhöhen.

Ein zentrales Entwicklungsfeld betrifft die Optimierung der Dialogführung und der Prozessdurchlaufzeiten. Die schrittweise Erfassung zahlreicher Datenpunkte, insbesondere bei der Neuanlage von Kunden, führt zu vergleichsweise langen und potenziell ermüdenden Interaktionen. Künftige Arbeiten könnten hier ansetzen, indem Eingaben stärker gebündelt, geeignete Voreinstellungen genutzt oder bestehende Informationen aus früheren Interaktionen wiederverwendet werden. Ergänzend bietet sich eine weitergehende Validierung von Eingaben sowie eine verbesserte Fehlertoleranz an, um Sonderfälle robuster abzufangen und Unterbrechungen im Ablauf zu vermeiden.

Darüber hinaus eröffnet die Integration mit weiteren Systemen und Datenquellen zusätzliche Potenziale. Denkbar sind etwa Schnittstellen zu Buchhaltungs- oder Steuersoftware, um erzeugte Rechnungen automatisiert zu übergeben, oder der Zugriff auf Stammdaten aus bestehenden Kundensystemen, um Doppeleingaben zu vermeiden. Auch eine Erweiterung der Lösung auf weitere Dokumenttypen und administrative Prozesse – beispielsweise Angebote, Mahnungen oder vertragsbezogene Unterlagen – könnte den Nutzen des Ansatzes für kleine Unternehmen erhöhen und zu einem umfassenderen, modular erweiterbaren Administrationswerkzeug führen.

Schließlich stellen Skalierbarkeit, Betriebssicherheit und langfristige Wartbarkeit wichtige Themen für eine produktive Nutzung dar. Zukünftige Arbeiten könnten untersuchen, wie sich Abhängigkeiten von einzelnen Cloud-Diensten reduzieren, Monitoring- und Logging-Konzepte ausbauen und Sicherheits- sowie Datenschutzanforderungen systematisch adressieren lassen. Auf dieser Grundlage wäre es möglich, den in dieser Arbeit entwickelten Prototyp schrittweise von einer experimentellen Lösung zu einem belastbaren System weiterzuentwickeln, das in der Praxis dauerhaft eingesetzt werden kann.

