% first example chapter
% @author Thomas Lehmann
%
\chapter{Theoretische Grundlagen}
\section{Digitalisierung administrativer Prozesse und Rechnungswesen}

Die vorliegende Arbeit richtet sich nicht an Großunternehmen mit etablierten ERP- und Buchhaltungssystemen. Im Fokus stehen vielmehr kleine Dienstleistungsunternehmen und Einzelunternehmer, die ihre administrativen Aufgaben überwiegend ohne eigene Buchhaltungsabteilung bewältigen. Zu dieser Zielgruppe zählen insbesondere Freelancer sowie Kleinstbetriebe wie Handwerker, Berater, Coaches oder Fotografen. Darüber hinaus werden auch kleine Agenturen und Dienstleistungsbetriebe mit wenigen Mitarbeitenden berücksichtigt. In diesen Unternehmen stützt sich das Rechnungswesen häufig auf einfache Office-Werkzeuge und wenig strukturierte Ablagesysteme. Viele administrative Tätigkeiten werden manuell durchgeführt und binden einen erheblichen Teil der verfügbaren Arbeitszeit. Für diese Zielgruppe stellt die Digitalisierung administrativer Prozesse eine zentrale Chance dar. Sie ermöglicht es, den manuellen Aufwand im agesgeschäft zu reduzieren und gleichzeitig die Qualität sowie Nachvollziehbarkeit finanzieller Informationen nachhaltig zu verbessern.

Im Alltag nutzen kleine Unternehmen häufig Word- oder Excel-Vorlagen zur Erstellung von Angeboten und Rechnungen. Bestehende Dokumente werden dabei kopiert und manuell angepasst. Kundendaten, Rechnungsnummern, Datumsangaben sowie Beträge und Steuerinformationen müssen wiederholt per Hand eingetragen werden. Dieses Vorgehen ist zeitaufwendig und fehleranfällig. Aufgrund fehlender Systemunterstützung können Tippfehler, unvollständige Angaben oder doppelt vergebene Rechnungsnummern entstehen. Die Kommunikation mit Kunden und Steuerberatungen erfolgt zudem überwiegend per E-Mail, wodurch häufig mehrere Versionen derselben Rechnung im Umlauf sind und zusätzlicher Abstimmungsaufwand entsteht. Belege liegen darüber hinaus in unterschiedlichen Formaten vor, etwa als Papierdokumente, Scans, Smartphone-Fotos oder PDF-Anhänge. Die Ablage erfolgt häufig unsystematisch in lokalen Ordnerstrukturen, Cloud-Speichern oder direkt im E-Mail-Postfach. Ein einheitliches Ordnungsprinzip nach Kunden oder Zeiträumen fehlt dabei oftmals. Diese Arbeitsweise ist von zahlreichen Medienbrüchen geprägt, da Informationen manuell zwischen verschiedenen Anwendungen und Dokumenten übertragen werden müssen. Jeder Wechsel zwischen Medien und Systemen erfordert zusätzliche manuelle Eingriffe und erhöht die Fehleranfälligkeit. Gleichzeitig nimmt die Transparenz über den aktuellen Status von Rechnungen und Belegen ab. Insbesondere vor periodischen Stichtagen wie Monats- oder Jahresabschlüssen führt diese Situation zu zusätzlichem organisatorischem Aufwand und erhöhtem Stress, da Unterlagen für Steuerberaterinnen und Steuerberater häufig nachträglich zusammengestellt werden müssen. Verzögerte oder fehlerhafte Rechnungen wirken sich zudem unmittelbar auf die Liquidität aus, da Zahlungseingänge verspätet erfolgen und offene Posten schwerer nachzuvollziehen sind.

Die Digitalisierung administrativer Prozesse im Rechnungswesen verfolgt mehrere zentrale Ziele. Ein wesentliches Ziel ist die Verkürzung von Durchlaufzeiten. Wiederkehrende Prozessschritte wie das Erstellen,Versenden und Ablegen von Rechnungen sollen hierzu standardisiert und weitgehend automatisiert ablaufen. Darüber hinaus soll die Nachvollziehbarkeit verbessert werden. Rechnungen und zugehörige Dokumente werden zentral abgelegt und eindeutig Kunden sowie Zeiträumen zugeordnet. Dadurch sind sie jederzeit auswertbar, und die Transparenz über den Status von Belegen und Zahlungsvorgängen wird erhöht. Ein weiterer Schwerpunkt liegt auf der Reduktion redundanter Dateneingaben. Kundendaten, Leistungsbeschreibungen und Zahlungsinformationen sollen nur einmal strukturiert erfasst und anschließend in unterschiedlichen Kontexten wiederverwendet werden.Dazu zählen unter anderem die Rechnungserstellung, Auswertungen und die Vorbereitung für die Steuerberatung. Cloudbasierte Lösungen ermöglichen zudem einen orts- und zeitunabhängigen Zugriff auf Rechnungen und Belege. Dadurch werden sowohl die Zusammenarbeit mit externen Dienstleistern als auch mobile Arbeitsformen erleichtert.

In Bezug auf das Szenario sind vor allem die Informationen entscheidend, die nötig sind, um eine korrekte und nachvollziehbare Rechnungserstellung zu gewährleisten. Hierzu gehören konsistente Kundendaten wie Adresse und Umsatzsteuer-Identifikationsnummer, eindeutige Rechnungsnummern und Rechnungsdaten sowie strukturierte Leistungsbeschreibungen mit Mengen- und Preisangaben.  Darüber hinaus sind steuerliche Kenngrößen und klar definierte Zahlungsbedingungen von Bedeutung. Das System, das in dieser Arbeit entworfen wurde, adressiert diese Anforderungen bewusst als Vorstufe zur Finanzbuchhaltung. Es ermöglicht die strukturierte Erfassung von Rechnungsdaten, die zentrale Verwaltung von Kundenstammdaten, die automatisierte Erstellung von Dokumenten und eine revisionssichere Ablage der erzeugten Belege. Es werden weder Hauptbücher geführt noch Buchungssätze erstellt. Aufgaben wie die Kontierung, die Umsatzsteuervoranmeldung oder der Jahresabschluss bleiben nach wie vor bei Steuerberatungen und spezialisierten Buchhaltungssystemen. Sie können auf den konsistenten Beleg- und Stammdaten aufbauen, die das System bereitstellt, und die weiterführende buchhalterische Verarbeitung übernehmen.

Der in dieser Arbeit verfolgte Chatbot-basierte Ansatz unterscheidet sich grundlegend von klassischen, formularorientierten Rechnungsprogrammen, wie sie typischerweise über webbasierte Benutzeroberflächen bereitgestellt werden. Herkömmliche Systeme setzen voraus, dass Nutzer aktiv eine Anwendung öffnen, sich durch Masken und Menüs navigieren und eigenständig entscheiden, welche Felder auszufüllen sind. Der hier vorgestellte Ansatz verfolgt hingegen eine schrittweise, dialogbasierte Datenerfassung über einen bereits etablierten Kommunikationskanal. Nutzer interagieren mit dem System ähnlich wie mit einem Kontakt in einem Messenger und werden im Gesprächsverlauf gezielt durch den Erfassungsprozess geführt. Die selbstständige Bedienung komplexer Formulare ist dadurch nicht erforderlich. Dieser Ansatz senkt insbesondere für technisch weniger affine Anwender die Einstiegshürde. Zudem ermöglicht er eine situative Datenerfassung direkt im Arbeitskontext, etwa unmittelbar nach Abschluss einer Dienstleistung. Geführte Eingaben und integrierte Validierungen tragen zusätzlich dazu bei, die Fehleranfälligkeit zu reduzieren. Studien zur Automatisierung von Finanzprozessen mit Chatbots zeigen, dass dialogbasierte Systeme insbesondere bei der Verarbeitung von Rechnungs- und Ausgabendaten signifikante Effizienzgewinne erzielen und gleichzeitig die Fehlerquote verringern können.



\section{Chatbots und dialogbasierte Systeme}
\section{Cloudbasierte Datenhaltung und Integrationen}
\section{Dokumentengenerierung in der Cloud}
\section{Grundlagen moderner Webanwendungen}