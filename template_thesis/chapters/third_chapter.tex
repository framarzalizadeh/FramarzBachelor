% first example chapter
% @author Thomas Lehmann
%
\chapter{Technologischer Hintergrund}
\section{Make.com als Integrations- und Automatisierungsplattform}
\label{sec:make}

Make.com ist eine cloudbasierte Integrations- und Automatisierungsplattform,
die es ermöglicht, unterschiedliche Anwendungen, Dienste und Systeme
miteinander zu verbinden. Die Umsetzung erfolgt ohne klassische
Programmierung und basiert auf einem No-Code- beziehungsweise
Low-Code-Ansatz. Workflows werden in einer grafischen Benutzeroberfläche modelliert, indem vordefinierte Funktionsbausteine konfiguriert und zu logischen Abläufen
verknüpft werden. Dadurch können auch komplexe Integrationsprozesse ohne
umfangreiche Programmierkenntnisse umgesetzt werden.Aus technischer Sicht handelt es sich bei Make.com um eine Integrationsplattform, die externe Systeme über deren Web-APIs anbindet. Ereignisse und Datenflüsse werden dabei in einem zentralen
Orchestrierungslayer zusammengeführt und automatisiert verarbeitet.

Das zentrale Konstrukt in Make sind sogenannte Szenarien. Ein Szenario
beschreibt einen vollständigen Automatisierungsablauf, der aus einem
auslösenden Ereignis (Trigger), einer Abfolge von Verarbeitungsschritten
sowie einer oder mehreren Aktionen besteht.Die einzelnen Bausteine eines Szenarios werden als Module bezeichnet.
Diese repräsentieren entweder konkrete Anwendungsintegrationen, etwa zu
WhatsApp, Airtable oder Google Docs, oder generische Funktionen wie Filter,
Iteratoren, Aggregatoren oder Router. Trigger-Module starten ein Szenario
beispielsweise bei einer eingehenden Nachricht über einen Webhook. Aktionsmodule übernehmen anschließend die Verarbeitung der Daten, indem
sie Informationen lesen, schreiben oder an andere Systeme übertragen.
Mithilfe visueller Router und Filter können alternative Verarbeitungswege
modelliert werden, die abhängig von Dateninhalten oder Zuständen
unterschiedliche Pfade innerhalb des Szenarios durchlaufen.

Die Plattform stellt darüber hinaus erweiterte Funktionen zur
Ablaufsteuerung bereit. Mithilfe von Filtern können nur jene Datensätze
weiterverarbeitet werden, die definierte Bedingungen erfüllen, etwa das
Vorhandensein bestimmter Pflichtfelder. Iteratoren und Aggregatoren ermöglichen die Verarbeitung von Listenstrukturen. Dabei können Datensätze entweder einzeln durchlaufen
oder zu zusammengefassten Ergebnissen aggregiert werden. Router-Module
erlauben es, Workflows in mehrere Pfade zu verzweigen, die jeweils eigene
Bedingungen und nachgelagerte Verarbeitungsschritte besitzen. Auf diese Weise lassen sich auch komplexe und verzweigte Prozesse
modellieren, ohne dass eigener Kontrollfluss-Code implementiert werden
muss. Eine besondere Rolle nehmen Webhook-Trigger ein, da sie die nahezu
Echtzeit-Verarbeitung von Ereignissen aus externen Systemen ermöglichen,
sobald diese eine definierte URL aufrufen.

Im betrachteten Anwendungskontext fungiert Make.com als zentrale
Vermittlungsschicht zwischen Chatbot, Datenhaltung sowie Dokumenten- und
Speicherdiensten. Ereignisse aus der Kommunikationsschnittstelle werden
über Trigger-Module empfangen, innerhalb von Szenarien verarbeitet und
anschließend über Anwendungs-Module an Systeme wie Airtable, Google Docs
oder Google Drive weitergeleitet.Die Plattform übernimmt dabei die Steuerung der Datenflüsse, führt Validierungen aus und stößt nachgelagerte Aktionen an, etwa das Erzeugen
von Dokumenten oder das Aktualisieren von Datensätzen. Make.com erfüllt
damit die Rolle einer zentralen Orchestrierungsebene innerhalb der
Systemarchitektur. Insbesondere für kleinere Projekte und prototypische Anwendungen ergeben
sich daraus mehrere Vorteile. Die initiale Implementierung kann ohne
eigene Serverinfrastruktur erfolgen, vorgefertigte Konnektoren reduzieren
den Integrationsaufwand, und die Prozesslogik lässt sich durch die visuelle
Modellierung nachvollziehbar dokumentieren.

Gleichzeitig sind mit dem Einsatz einer Integrationsplattform wie Make.com
auch Einschränkungen verbunden. Sehr umfangreiche Szenarien können an
Übersichtlichkeit verlieren, da sämtliche Verarbeitungswege innerhalb
eines visuellen Diagramms abgebildet werden müssen. Zudem ist die
Abbildung komplexer Geschäftslogik nur eingeschränkt möglich. Darüber hinaus entsteht eine Abhängigkeit vom jeweiligen Anbieter, etwa
hinsichtlich Verfügbarkeit, Preisgestaltung und Funktionsumfang der
Plattform. Für hochskalierende oder besonders sicherheitskritische
Anwendungen kann daher eine klassisch implementierte, individuell
entwickelte Integrationsschicht sinnvoller sein. Make.com stellt hingegen insbesondere für klein- bis mittelkomplexe
Automatisierungsvorhaben mit begrenzten Ressourcen eine pragmatische und
gut nachvollziehbare Lösung dar.


\section{Airtable als Datenhaltung}
Airtable ist eine cloudbasierte Plattform, die tabellenorientierte
Benutzeroberflächen mit den Eigenschaften einer datenbankgestützten
Anwendung kombiniert. Aus Anwendersicht ähnelt Airtable einer klassischen
Tabellenkalkulation, erweitert diese jedoch um strukturierte Feldtypen,
Relationen zwischen Tabellen sowie eine standardisierte
Programmierschnittstelle. Dadurch eignet sich die Plattform insbesondere für Anwendungsszenarien, in denen Daten sowohl über eine webbasierte Oberfläche gepflegt als auch automatisiert von externen Systemen gelesen und geschrieben werden sollen.

Zentrale Konzepte in Airtable sind sogenannte Bases, die jeweils eine
in sich geschlossene Datenbankinstanz mit mehreren Tabellen
repräsentieren. Eine Tabelle besteht aus Datensätzen und Feldern, wobei
die Felder unterschiedliche Typen annehmen können, etwa Text, Zahl,
Datum, Auswahlfelder oder Verknüpfungen zu anderen Tabellen. Über verknüpfte Felder lassen sich einfache Relationen zwischen Tabellen modellieren, beispielsweise zwischen Kunden und zugehörigen Rechnungen. Darüber hinaus stehen verschiedene Sichten und Filter zur Verfügung, mit denen Teilmengen der Daten anhand frei definierbarer Kriterien angezeigt werden können. Formelfelder ermöglichen zudem die Berechnung dynamischer Werte. Sie können unter anderem zur Aggregation von Beträgen oder zur Erzeugung
formatierter Anzeigeinhalte eingesetzt werden.

Für den Einsatz als zentrale Datenhaltung in einem verteilten System ist
insbesondere die von Airtable bereitgestellte REST-API von Bedeutung. Sie
ermöglicht es, Datensätze programmatisch anzulegen, zu aktualisieren, zu
löschen oder anhand definierter Filterkriterien abzufragen. Auf diese Weise kann Airtable als zentrale „Single Source of Truth“ für fachliche Informationen fungieren. Unterschiedliche Frontends und Automatisierungsprozesse greifen dabei über die API auf denselben
konsistenten Datenbestand zu. Durch die Kombination aus grafischer Benutzeroberfläche und Programmierschnittstelle eignet sich Airtable sowohl für die manuelle
Pflege und Kontrolle von Daten als auch für die Integration in
automatisierte Workflows.

Im Kontext kleiner, cloudbasierter Anwendungen bietet Airtable mehrere
Vorteile gegenüber klassischen Datenbanksystemen. Die Tabellenstruktur
kann ohne aufwendige Deployment- oder Migrationsprozesse angepasst
werden, was insbesondere iteratives Arbeiten und prototypische
Entwicklungen erleichtert. Darüber hinaus unterstützt die grafische Benutzeroberfläche die manuelle Datenpflege sowie die Fehlersuche. Datensätze sind unmittelbar einsehbar
und bearbeitbar, wodurch einfache Auswertungen und Kontrollen direkt in
der Anwendung durchgeführt werden können. Demgegenüber stehen typische Einschränkungen gehosteter Plattformen. Dazu zählen unter anderem Begrenzungen hinsichtlich des maximalen Datenvolumens, der Anzahl zulässiger API-Aufrufe sowie die fehlende Unterstützung komplexer Transaktionen. Für klein- bis mittelkomplexe Anwendungen mit moderatem Datenumfang stellt Airtable dennoch eine pragmatische Lösung dar. Es ermöglicht die Realisierung einer zentralen, cloudbasierten Datenhaltung, ohne dass eine eigene Datenbankinfrastruktur betrieben werden muss.

\section{Google Docs und Google Drive}
Google Docs und Google Drive sind zentrale Bestandteile der
Google-Workspace-Plattform und bilden gemeinsam eine cloudbasierte
Umgebung zur Erstellung, Bearbeitung und Ablage von Dokumenten. Google
Docs stellt einen webbasierten Texteditor bereit, der die gleichzeitige
Bearbeitung eines Dokuments durch mehrere Personen ermöglicht.
Google Drive fungiert als zentraler Dateispeicher und bietet strukturierte
Ordner, Freigabefunktionen sowie Versionsverwaltung. Beide Dienste sind
vollständig browserbasiert und erlauben einen orts- und
geräteunabhängigen Zugriff auf Dokumente, ohne dass eine lokale
Office-Software installiert werden muss.




Für automatisierte Systeme ist insbesondere die Programmierschnittstelle
von Google Docs von Bedeutung. Sie ermöglicht es, Dokumente nicht nur
manuell, sondern auch programmatisch zu erzeugen und zu verändern. In der Praxis werden hierzu häufig Dokumentvorlagen verwendet, die das
Layout sowie feste Textbestandteile definieren. Variable Inhalte werden
über Platzhalter integriert, die bei der Dokumentenerstellung durch
konkrete Werte aus einem Datensystem ersetzt werden.
Auf diese Weise können aus einer einzelnen Vorlage eine Vielzahl
individueller Dokumente generiert werden. Typische Anwendungsfälle sind
unter anderem die standardisierte Erstellung von Rechnungen, Angeboten
oder Serienbriefen, ohne dass jedes Dokument manuell angepasst werden
muss.



Google Drive ergänzt diese Funktionalität durch eine strukturierte,
cloudbasierte Ablage der erzeugten Dokumente. Dateien können in
hierarchischen Ordnerstrukturen organisiert, mit Metadaten versehen und
über differenzierte Freigabemechanismen geteilt werden. Über die Google-Drive-API lassen sich Ordner und Dateien zudem programmgesteuert anlegen, verschieben, umbenennen dder herunterladen. Dadurch können Dokumentenerzeugung und Archivierung in einen
durchgängigen digitalen Prozess integriert werden. Insbesondere für kleinere Organisationen stellt Google Drive damit eine leichtgewichtige Alternative zu klassischen Dokumentenmanagementsystemen dar. Sowohl die manuelle Nutzung über die Weboberfläche als auch automatisierte Workflows auf Basis standardisierter Schnittstellen werden dabei unterstützt.




\section{Lovable Cloud / Supabase Edge Functions}
\section{Frontend-Stack (React, TypeScript, Tailwind, shadcn/ui)}